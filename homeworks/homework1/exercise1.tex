% !TEX TS-program = pdflatex
% !TEX encoding = UTF-8 Unicode

% This is a simple template for a LaTeX document using the "article" class.
% See "book", "report", "letter" for other types of document.

\documentclass[11pt]{article} % use larger type; default would be 10pt

\usepackage[utf8]{inputenc} % set input encoding (not needed with XeLaTeX)
\usepackage{hyperref}
\usepackage{color}

%%% Examples of Article customizations
% These packages are optional, depending whether you want the features they provide.
% See the LaTeX Companion or other references for full information.

%%% PAGE DIMENSIONS
\usepackage{geometry} % to change the page dimensions
\geometry{a4paper} % or letterpaper (US) or a5paper or....
% \geometry{margin=2in} % for example, change the margins to 2 inches all round
% \geometry{landscape} % set up the page for landscape
%   read geometry.pdf for detailed page layout information

\usepackage{graphicx} % support the \includegraphics command and options

% \usepackage[parfill]{parskip} % Activate to begin paragraphs with an empty line rather than an indent

%%% PACKAGES
\usepackage{booktabs} % for much better looking tables
\usepackage{array} % for better arrays (eg matrices) in maths
\usepackage{paralist} % very flexible & customisable lists (eg. enumerate/itemize, etc.)
\usepackage{verbatim} % adds environment for commenting out blocks of text & for better verbatim
\usepackage{subfig} % make it possible to include more than one captioned figure/table in a single float
% These packages are all incorporated in the memoir class to one degree or another...

%%% HEADERS & FOOTERS
\usepackage{fancyhdr} % This should be set AFTER setting up the page geometry
\pagestyle{fancy} % options: empty , plain , fancy
\renewcommand{\headrulewidth}{0pt} % customise the layout...
\lhead{}\chead{}\rhead{}
\lfoot{}\cfoot{\thepage}\rfoot{}

%%% SECTION TITLE APPEARANCE
\usepackage{sectsty}
\allsectionsfont{\sffamily\mdseries\upshape} % (See the fntguide.pdf for font help)
% (This matches ConTeXt defaults)

%%% ToC (table of contents) APPEARANCE
\usepackage[nottoc,notlof,notlot]{tocbibind} % Put the bibliography in the ToC
\usepackage[titles,subfigure]{tocloft} % Alter the style of the Table of Contents
\renewcommand{\cftsecfont}{\rmfamily\mdseries\upshape}
\renewcommand{\cftsecpagefont}{\rmfamily\mdseries\upshape} % No bold!

%%% END Article customizations

%%% The "real" document content comes below...

\title{Exercise 0: Setting up Python and Jupyter}
%\date{} % Activate to display a given date or no date (if empty),
         % otherwise the current date is printed 

\begin{document}
\maketitle

\section{Installing anaconda}

In this course, you will be asked to implement code in Python. In order to ensure compatability of your code, we need to ensure that we are all working with the same Python version. We will work with a pre-packaged Python distribution called `Anaconda'.\\

Follow the instructions on \textcolor{blue}{\href{https://www.continuum.io/downloads}{here}} and install Anaconda with the \textbf{Python version 3.5}(!!).


\section{Installing additional packages}
A very handy package for data analysis is `sklearn' (scikit-learn). Follow  \textcolor{blue}{\href{http://conda.pydata.org/docs/test-drive.html\#managing-packages}{these instructions}} to install the package \emph{scikit-learn}.

\newpage
\section{Jupyter Notebooks}
The Jupyter Notebook is a web application that allows you to create and share documents that contain live code, equations, visualizations and explanatory text. You will be asked to hand in your homeworks as Jupyter notebooks. Let's make sure you have a basic understanding of notebooks. \\

Create a directory where you are planning on putting your homeworks. Open a terminal and navigate to that directory. Check out the  \textcolor{blue}{\href{https://github.com/marioberges/F16-12-752}{git-repository}} for this course. Consult the  \textcolor{blue}{\href{https://services.github.com/kit/downloads/github-git-cheat-sheet.pdf}{git cheat sheet}} if you need help using git. Once you have successfully cloned the repository, open a terminal (or command prompt depending on your OS) and run the command ``jupyter notebook''. Your web browser should now open and the Jupyter webpage should be displayed. Select the Jupyter notebook for the first homework (exercise0.ipnb) and open it in the web interface. Follow the instructions in the notebook. In the end, save the notebook and put it on github.


\end{document}
